\documentclass[14pt]{extarticle}

\input src/headers.tex
\input src/commands.tex
\input src/styles.tex

\title{\vspace{-4ex}\Large{C++ Snippets}}
\author{Muhammad Samir Assawalhy}
\date{\today}

\pagestyle{fancy}
\fancyhead[L]{Assawalhy's Snippets}
\fancyhead[R]{\thepage}
\fancyfoot[C]{}

\fancypagestyle{plain} {
  \fancyhead[L]{Assawalhy's Snippets}
  \fancyhead[R]{\thepage}
  \fancyfoot[C]{}
}

\begin{document}

\maketitle
\vspace{-13ex}

\tableofcontents

\pagestyle{fancy}

% \input src/sections.tex

\section{Math & Miscellaneous}

\subsection{Master's theorem}

Consider the recurrence relation:
\[
T(n) = a T\left(\frac{n}{b}\right) + f(n)
\]
where \( a \geq 1 \) and \( b > 1 \). The solution to this recurrence relation depends on the asymptotic behavior of \( f(n) \) compared to \( n^{\log_b a} \):

\begin{enumerate}
    \item If \( f(n) = O(n^c) \) where \( c < \log_b a \), then:
    \[
    T(n) = O(n^{\log_b a})
    \]
    \item If \( f(n) = \Theta(n^{\log_b a}) \), then:
    \[
    T(n) = \Theta(n^{\log_b a} \log n)
    \]
    \item If \( f(n) = \Omega(n^c) \) where \( c > \log_b a \), and if \( a f\left(\frac{n}{b}\right) \leq k f(n) \) for some constant \( k < 1 \) and sufficiently large \( n \), then:
    \[
    T(n) = \Theta(f(n))
    \]
\end{enumerate}

\subsection{Euler Totient $\phi(n)$}

\begin{itemize}
  \item $\phi(n)$ is the count of co-primes of $n$ from $1$ to $n$.
  \item So that $\phi(4) = 2$ which are $3$ and $1$.
  \item $\phi(p) = p - 1$ for a prime $p$.
\end{itemize}

The following is Euler's theorem which can be used to compute modular inverse of $a$ only if $\gcd(a, m) = 1$.

\[
  a^{\phi(m)} \equiv 1 \mod m
\]

When $g = \gcd(a, m) > 1$ and you want to compute $a^x$ when $x$ is too large to do with binary exponentiation:

\begin{align*}
    a^{\phi(m)} \not\equiv 1 &\pmod{m} \\
    a^{x} \equiv \left(\frac{a}{g}\right)^x \cdot g^x &\pmod{m} \\
    a^{\phi\left(\frac m g\right)} \equiv g^{\phi\left(\frac m g\right)} \equiv 1 &\pmod{\frac{m}{g}} \\
    g \cdot g^{\phi\left(\frac m g\right)} \equiv g &\pmod{m} \\
    a^{x} \equiv \left(\frac{a}{g}\right)^{x \bmod m} \cdot g^x &\pmod{m}
\end{align*}

To compute $g^x \bmod m$ we should handle two cases:

\begin{enumerate}
  \item When $x \lte m$ we can simply do it using binary expo.
  \item When $x \ge m$:
    \begin{itemize}
      \item 
    \end{itemize}
\end{enumerate}


\subsection{Modular arithmetics}

Get inverse of number module prime $p$ for all values in $[1, x]$:

\begin{algorithm}
    \begin{algorithmic}[1]
        \State Initialize $inv$ array with $1$s of size $x + 1$
        \For{$i \gets 2$ to $x$}
            \State $inv_i \gets - \frac p i \cdot inv_{p \bmod i} \mod p$
        \EndFor
    \end{algorithmic}
\end{algorithm}



\end{document}
