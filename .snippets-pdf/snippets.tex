\documentclass[14pt]{extarticle}

\input src/headers.tex
\input src/commands.tex
\input src/styles.tex

\title{\vspace{-4ex}\Large{C++ Snippets}}
\author{Muhammad Samir Assawalhy}
\date{\today}

\pagestyle{fancy}
\fancyhead[L]{Assawalhy's Snippets}
\fancyhead[R]{\thepage}
\fancyfoot[C]{}

\fancypagestyle{plain} {
  \fancyhead[L]{Assawalhy's Snippets}
  \fancyhead[R]{\thepage}
  \fancyfoot[C]{}
}

\begin{document}

\maketitle
\vspace{-13ex}

\tableofcontents

\pagestyle{fancy}

\input src/sections.tex

\section{Math & Miscellaneous}

\subsection{Master's theorem}

Consider the recurrence relation:
\[
T(n) = a T\left(\frac{n}{b}\right) + f(n)
\]
where \( a \geq 1 \) and \( b > 1 \). The solution to this recurrence relation depends on the asymptotic behavior of \( f(n) \) compared to \( n^{\log_b a} \):

\begin{enumerate}
    \item If \( f(n) = O(n^c) \) where \( c < \log_b a \), then:
    \[
    T(n) = O(n^{\log_b a})
    \]
    \item If \( f(n) = \Theta(n^{\log_b a}) \), then:
    \[
    T(n) = \Theta(n^{\log_b a} \log n)
    \]
    \item If \( f(n) = \Omega(n^c) \) where \( c > \log_b a \), and if \( a f\left(\frac{n}{b}\right) \leq k f(n) \) for some constant \( k < 1 \) and sufficiently large \( n \), then:
    \[
    T(n) = \Theta(f(n))
    \]
\end{enumerate}

\end{document}
